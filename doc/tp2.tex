\documentclass[12pt]{article}
\usepackage{sbc-template}
\usepackage{graphicx}
\usepackage[lofdepth,lotdepth]{subfig}
\usepackage{graphics}
\usepackage{amsmath}
\usepackage{wrapfig}
\usepackage{booktabs}
\usepackage{rotating}
\usepackage{times,amsmath,epsfig}
\usepackage{url}
\usepackage{multirow}
 \makeatletter
 \newif\if@restonecol
 \makeatother
 \let\algorithm\relax
 \let\endalgorithm\relax
\usepackage{listings}
\usepackage{float}
\usepackage[lined,algonl,ruled]{algorithm2e}
\usepackage{multirow}
\usepackage[brazil]{babel}
\usepackage[latin1]{inputenc}
\usepackage{enumitem}



% \setlist{nolistsep}

\sloppy

\title{Minera��o de Dados: Trabalho Pr�tico 2}

\author{Artur Rodrigues}

\address{Departamento de Ci�ncia da Computa��o \\ Universidade Federal de Minas Gerais (UFMG)
    \email{artur@dcc.ufmg.br}
}

\begin{document}

\maketitle

\section{INTRODU��O}


\section{K-MEANS}


\subsection{Complexidade}


\section{ESCOLHA DOS CENTR�IDES INICIAIS}

K random initial centroids
Forgy initialization - randomly chooses K observations from the data set
Random Partition - randomly assign a cluster to each observation and then proceeds to the Update step


\section{BASE DE DADOS}


\section{AVALIA��O EXPERIMENTAL}

\subsection{Procedimentos}

Com o intuito de se obter testes mais consistentes, os experimentos foram executados em ambiente virtualizado, com capacidade de processamento e mem�ria prim�ria reduzidas, 50\% da capacidade da m�quina hospedeira e 1024MiB, respectivamente. O sistema operacional do ambiente virtualizado era Ubuntu Server 12.04 64 bits e os softwares utilizados foram interpretador Python (2.7.2) PyPy vers�o 1.9.0, e GCC vers�o 4.2.1. A m�quina hospedeira possu�a sistema operacional Mac OS X 10.8.2, processador \textit{quad-core} de 2.3GHz e mem�ria prim�ria com capacidade de 16GiB.

Todos os testes foram realizados 5 vezes e o resultado m�dio para o tempo de execu��o foi considerado. Finalmente, certificou-se que a solu��o desenvolvida execute perfeitamente na esta��o \verb+claro.grad.dcc.ufmg.br+.


\subsection{An�lise do Tamanho da Entrada}


\subsection{An�lise de Par�metros}


\subsection{An�lise da Qualidade da Solu��o}

Smallest total squared distance


\section{CONCLUS�O}


% \nocite{*}
% \bibliographystyle{sbc}
% \bibliography{bib}

\end{document}
